\documentclass[aspectratio=43]{beamer}
\usepackage{Sweave}

\usetheme{default}
% Slide setup, colour independent

\usepackage{amsmath,amssymb,amsthm}
\usepackage[utf8]{inputenc}
\usepackage{colortbl}
\usepackage{bm}
\usepackage{xcolor}
\usepackage{dsfont}
\usepackage{setspace}
%\usepackage{subfigure}
% To use \ding{234} and the like
\usepackage{pifont}
% To cross reference between slide files
\usepackage{zref-xr,zref-user}
% Use something like
% \zexternaldocument{fileI}
% in the tex files. And cite using \zref instead of \ref

% Fields and the like
\def\IC{\mathbb{C}}
\def\IF{\mathbb{F}}
\def\II{\mathbb{I}}
\def\IJ{\mathbb{J}}
\def\IM{\mathbb{M}}
\def\IN{\mathbb{N}}
\def\IP{\mathbb{P}}
\def\IR{\mathbb{R}}
\def\IZ{\mathbb{Z}}
\def\11{\mathds{1}}


% Bold lowercase
\def\ba{\mathbf{a}}
\def\bb{\mathbf{b}}
\def\bc{\mathbf{c}}
\def\bd{\mathbf{d}}
\def\be{\mathbf{e}}
\def\bf{\mathbf{f}}
\def\bh{\mathbf{h}}
\def\bi{\mathbf{i}}
\def\bj{\mathbf{j}}
\def\bk{\mathbf{k}}
\def\bn{\mathbf{n}}
\def\bp{\mathbf{p}}
\def\br{\mathbf{r}}
\def\bs{\mathbf{s}}
\def\bu{\mathbf{u}}
\def\bv{\mathbf{v}}
\def\bw{\mathbf{w}}
\def\bx{\mathbf{x}}
\def\by{\mathbf{y}}
\def\bz{\mathbf{z}}

% Bold capitals
\def\bB{\mathbf{B}}
\def\bD{\mathbf{D}}
\def\bE{\mathbf{E}}
\def\bF{\mathbf{F}}
\def\bG{\mathbf{G}}
\def\bI{\mathbf{I}}
\def\bL{\mathbf{L}}
\def\bN{\mathbf{N}}
\def\bP{\mathbf{P}}
\def\bR{\mathbf{R}}
\def\bS{\mathbf{S}}
\def\bT{\mathbf{T}}
\def\bX{\mathbf{X}}

% Bold numbers
\def\b0{\mathbf{0}}

% Bold greek
\bmdefine{\bmu}{\bm{\mu}}
\def\bphi{\bm{\phi}}
\def\bvarphi{\bm{\varphi}}
\def\bPi{\bm{\Pi}}
\def\bGamma{\bm{\Gamma}}

% Bold red sentence
\def\boldred#1{{\color{red}\textbf{#1}}}
\def\defword#1{{\color{orange}\textbf{#1}}}

% Caligraphic letters
\def\A{\mathcal{A}}
\def\B{\mathcal{B}}
\def\C{\mathcal{C}}
\def\D{\mathcal{D}}
\def\E{\mathcal{E}}
\def\F{\mathcal{F}}
\def\G{\mathcal{G}}
\def\H{\mathcal{H}}
\def\I{\mathcal{I}}
\def\L{\mathcal{L}}
\def\M{\mathcal{M}}
\def\N{\mathcal{N}}
\def\P{\mathcal{P}}
\def\R{\mathcal{R}}
\def\S{\mathcal{S}}
\def\T{\mathcal{T}}
\def\U{\mathcal{U}}
\def\V{\mathcal{V}}

% tt font for code
\def\code#1{{\tt #1}}

% i.e., e.g.
\def\eg{\emph{e.g.}}
\def\ie{\emph{i.e.}}


% Operators and special symbols
\def\nbOne{{\mathchoice {\rm 1\mskip-4mu l} {\rm 1\mskip-4mu l}
{\rm 1\mskip-4.5mu l} {\rm 1\mskip-5mu l}}}
\def\cov{\ensuremath{\mathsf{cov}}}
\def\Var{\ensuremath{\mathsf{Var}\ }}
\def\Im{\textrm{Im}\;}
\def\Re{\textrm{Re}\;}
\def\det{\ensuremath{\mathsf{det}}}
\def\diag{\ensuremath{\mathsf{diag}}}
\def\nullspace{\ensuremath{\mathsf{null}}}
\def\nullity{\ensuremath{\mathsf{nullity}}}
\def\rank{\ensuremath{\mathsf{rank}}}
\def\range{\ensuremath{\mathsf{range}}}
\def\sgn{\ensuremath{\mathsf{sgn}}}
\def\Span{\ensuremath{\mathsf{span}}}
\def\tr{\ensuremath{\mathsf{tr}}}
\def\imply{$\Rightarrow$}
\def\restrictTo#1#2{\left.#1\right|_{#2}}
\newcommand{\parallelsum}{\mathbin{\!/\mkern-5mu/\!}}
\def\dsum{\mathop{\displaystyle \sum }}%
\def\dind#1#2{_{\substack{#1\\ #2}}}

\DeclareMathOperator{\GL}{GL}
\DeclareMathOperator{\Rel}{Re}
\def\Nt#1{\left|\!\left|\!\left|#1\right|\!\right|\!\right|}
\newcommand{\tripbar}{|\! |\! |}



% The beamer bullet (in base colour)
\def\bbullet{\leavevmode\usebeamertemplate{itemize item}\ }

% Theorems and the like
\newtheorem{proposition}[theorem]{Proposition}
\newtheorem{property}[theorem]{Property}
\newtheorem{importantproperty}[theorem]{Property}
\newtheorem{importanttheorem}[theorem]{Theorem}
%\newtheorem{lemma}[theorem]{Lemma}
%\newtheorem{corollary}[theorem]{Corollary}
\newtheorem{remark}[theorem]{Remark}
\setbeamertemplate{theorems}[numbered]
%\setbeamertemplate{theorems}[ams style]

%
%\usecolortheme{orchid}
%\usecolortheme{orchid}

\def\red{\color[rgb]{1,0,0}}
\def\blue{\color[rgb]{0,0,1}}
\def\green{\color[rgb]{0,1,0}}


% Get rid of navigation stuff
\setbeamertemplate{navigation symbols}{}

% Set footline/header line
\setbeamertemplate{footline}
{%
\quad p. \insertpagenumber \quad--\quad \insertsection\vskip2pt
}
% \setbeamertemplate{headline}
% {%
% \quad\insertsection\hfill p. \insertpagenumber\quad\mbox{}\vskip2pt
% }


\makeatletter
\newlength\beamerleftmargin
\setlength\beamerleftmargin{\Gm@lmargin}
\makeatother

% Colours for special pages
\def\extraContent{yellow!20}


%%%%%%%%%%%%%%%%%
\usepackage{tikz}
\usetikzlibrary{shapes,arrows}
\usetikzlibrary{positioning}
\usetikzlibrary{shapes.symbols,shapes.callouts,patterns}
\usetikzlibrary{calc,fit}
\usetikzlibrary{backgrounds}
\usetikzlibrary{decorations.pathmorphing,fit,petri}
\usetikzlibrary{automata}
\usetikzlibrary{fadings}
\usetikzlibrary{patterns,hobby}

\usetikzlibrary{backgrounds,fit,petri}


\usepackage{pgfplots}
\pgfplotsset{compat=1.6}
\pgfplotsset{ticks=none}

\usetikzlibrary{decorations.markings}
\usetikzlibrary{arrows.meta}
\tikzset{>=stealth}

% For tikz
\usetikzlibrary{shapes,arrows}
\usetikzlibrary{positioning}
\tikzstyle{cloud} = [draw, ellipse,fill=red!20, node distance=0.87cm,
minimum height=2em]
\tikzstyle{line} = [draw, -latex']


%%% For max frame images
\newenvironment{changemargin}[2]{%
\begin{list}{}{%
\setlength{\topsep}{0pt}%
\setlength{\leftmargin}{#1}%
\setlength{\rightmargin}{#2}%
\setlength{\listparindent}{\parindent}%
\setlength{\itemindent}{\parindent}%
\setlength{\parsep}{\parskip}%
}%
\item[]}{\end{list}}


% Make one image take up the entire slide content area in beamer,.:
% centered/centred full-screen image, with title:
% This uses the whole screen except for the 1cm border around it
% all. 128x96mm
\newcommand{\titledFrameImage}[2]{
\begin{frame}{#1}
%\begin{changemargin}{-1cm}{-1cm}
\begin{center}
\includegraphics[width=108mm,height=\textheight,keepaspectratio]{#2}
\end{center}
%\end{changemargin}
\end{frame}
}

% Make one image take up the entire slide content area in beamer.:
% centered/centred full-screen image, no title:
% This uses the whole screen except for the 1cm border around it
% all. 128x96mm
\newcommand{\plainFrameImage}[1]{
\begin{frame}[plain]
%\begin{changemargin}{-1cm}{-1cm}
\begin{center}
\includegraphics[width=108mm,height=76mm,keepaspectratio]{#1}
\end{center}
%\end{changemargin}
\end{frame}
}

% Make one image take up the entire slide area, including borders, in beamer.:
% centered/centred full-screen image, no title:
% This uses the entire whole screen
\newcommand{\maxFrameImage}[1]{
\begin{frame}[plain]
\begin{changemargin}{-1cm}{-1cm}
\begin{center}
\includegraphics[width=\paperwidth,height=\paperheight,keepaspectratio]
{#1}
\end{center}
\end{changemargin}
\end{frame}
}

% This uses the entire whole screen (to include in frame)
\newcommand{\maxFrameImageNoFrame}[1]{
\begin{changemargin}{-1cm}{-1cm}
\begin{center}
\includegraphics[width=\paperwidth,height=0.99\paperheight,keepaspectratio]
{#1}
\end{center}
\end{changemargin}
}

% Make one image take up the entire slide area, including borders, in beamer.:
% centered/centred full-screen image, no title:
% This uses the entire whole screen
\newcommand{\maxFrameImageColor}[2]{
\begin{frame}[plain]
\setbeamercolor{normal text}{bg=#2!20}
\begin{changemargin}{-1cm}{-1cm}
\begin{center}
\includegraphics[width=\paperwidth,height=\paperheight,keepaspectratio]
{#1}
\end{center}
\end{changemargin}
\end{frame}
}


\usepackage{tikz}
\usetikzlibrary{patterns,hobby}
\usepackage{pgfplots}
\pgfplotsset{compat=1.6}
\pgfplotsset{ticks=none}

\usetikzlibrary{backgrounds}
\usetikzlibrary{decorations.markings}
\usetikzlibrary{arrows.meta}
\tikzset{>=stealth}

\tikzset{
  clockwise arrows/.style={
    postaction={
      decorate,
      decoration={
        markings,
        mark=between positions 0.1 and 0.9 step 40pt with {\arrow{>}},
   }}}}


   %%%%%%%%%%%
% To have links to parts in the outline
\makeatletter
\AtBeginPart{%
  \addtocontents{toc}{\protect\beamer@partintoc{\the\c@part}{\beamer@partnameshort}{\the\c@page}}%
}
%% number, shortname, page.
\providecommand\beamer@partintoc[3]{%
  \ifnum\c@tocdepth=-1\relax
    % requesting onlyparts.
    \makebox[6em]{Part #1:} \textcolor{green!30!blue}{\hyperlink{#2}{#2}}
    \par
  \fi
}
\define@key{beamertoc}{onlyparts}[]{%
  \c@tocdepth=-1\relax
}
\makeatother%

\newcommand{\nameofthepart}{}
\newcommand{\nupart}[1]%
    {   \part{#1}%
        \renewcommand{\nameofthepart}{#1}%
        {
          \setbeamercolor{background canvas}{bg=orange!50}
          \begin{frame}{#1}%\partpage 
          \hypertarget{\nameofthepart}{}\tableofcontents%
          \end{frame}
        }
    }



\usecolortheme{orchid}
%% Listings
\usepackage{listings}
\definecolor{mygreen}{rgb}{0,0.6,0}
\definecolor{mygray}{rgb}{0.5,0.5,0.5}
\definecolor{mymauve}{rgb}{0.58,0,0.82}
\definecolor{mygold}{rgb}{1,0.843,0}
\definecolor{myblue}{rgb}{0.537,0.812,0.941}

\definecolor{lgreen}{rgb}{0.6,0.9,.6}
\definecolor{lred}{rgb}{1,0.5,.5}

\lstloadlanguages{R}
\lstset{ %
  language=R,
  backgroundcolor=\color{black!05},   % choose the background color
  basicstyle=\footnotesize\ttfamily,        % size of fonts used for the code
  breaklines=true,                 % automatic line breaking only at whitespace
  captionpos=b,                    % sets the caption-position to bottom
  commentstyle=\color{mygreen},    % comment style
  escapeinside={\%*}{*)},          % if you want to add LaTeX within your code
  keywordstyle=\color{red},       % keyword style
  stringstyle=\color{mygold},     % string literal style
  keepspaces=true,
  columns=fullflexible,
  tabsize=4,
}
% Could also do (in lstset)
% basicstyle==\fontfamily{pcr}\footnotesize
\lstdefinelanguage{Renhanced}%
  {keywords={abbreviate,abline,abs,acos,acosh,action,add1,add,%
      aggregate,alias,Alias,alist,all,anova,any,aov,aperm,append,apply,%
      approx,approxfun,apropos,Arg,args,array,arrows,as,asin,asinh,%
      atan,atan2,atanh,attach,attr,attributes,autoload,autoloader,ave,%
      axis,backsolve,barplot,basename,besselI,besselJ,besselK,besselY,%
      beta,binomial,body,box,boxplot,break,browser,bug,builtins,bxp,by,%
      c,C,call,Call,case,cat,category,cbind,ceiling,character,char,%
      charmatch,check,chol,chol2inv,choose,chull,class,close,cm,codes,%
      coef,coefficients,co,col,colnames,colors,colours,commandArgs,%
      comment,complete,complex,conflicts,Conj,contents,contour,%
      contrasts,contr,control,helmert,contrib,convolve,cooks,coords,%
      distance,coplot,cor,cos,cosh,count,fields,cov,covratio,wt,CRAN,%
      create,crossprod,cummax,cummin,cumprod,cumsum,curve,cut,cycle,D,%
      data,dataentry,date,dbeta,dbinom,dcauchy,dchisq,de,debug,%
      debugger,Defunct,default,delay,delete,deltat,demo,de,density,%
      deparse,dependencies,Deprecated,deriv,description,detach,%
      dev2bitmap,dev,cur,deviance,off,prev,,dexp,df,dfbetas,dffits,%
      dgamma,dgeom,dget,dhyper,diag,diff,digamma,dim,dimnames,dir,%
      dirname,dlnorm,dlogis,dnbinom,dnchisq,dnorm,do,dotplot,double,%
      download,dpois,dput,drop,drop1,dsignrank,dt,dummy,dump,dunif,%
      duplicated,dweibull,dwilcox,dyn,edit,eff,effects,eigen,else,%
      emacs,end,environment,env,erase,eval,equal,evalq,example,exists,%
      exit,exp,expand,expression,External,extract,extractAIC,factor,%
      fail,family,fft,file,filled,find,fitted,fivenum,fix,floor,for,%
      For,formals,format,formatC,formula,Fortran,forwardsolve,frame,%
      frequency,ftable,ftable2table,function,gamma,Gamma,gammaCody,%
      gaussian,gc,gcinfo,gctorture,get,getenv,geterrmessage,getOption,%
      getwd,gl,glm,globalenv,gnome,GNOME,graphics,gray,grep,grey,grid,%
      gsub,hasTsp,hat,heat,help,hist,home,hsv,httpclient,I,identify,if,%
      ifelse,Im,image,\%in\%,index,influence,measures,inherits,install,%
      installed,integer,interaction,interactive,Internal,intersect,%
      inverse,invisible,IQR,is,jitter,kappa,kronecker,labels,lapply,%
      layout,lbeta,lchoose,lcm,legend,length,levels,lgamma,library,%
      licence,license,lines,list,lm,load,local,locator,log,log10,log1p,%
      log2,logical,loglin,lower,lowess,ls,lsfit,lsf,ls,machine,Machine,%
      mad,mahalanobis,make,link,margin,match,Math,matlines,mat,matplot,%
      matpoints,matrix,max,mean,median,memory,menu,merge,methods,min,%
      missing,Mod,mode,model,response,mosaicplot,mtext,mvfft,na,nan,%
      names,omit,nargs,nchar,ncol,NCOL,new,next,NextMethod,nextn,%
      nlevels,nlm,noquote,NotYetImplemented,NotYetUsed,nrow,NROW,null,%
      numeric,\%o\%,objects,offset,old,on,Ops,optim,optimise,optimize,%
      options,or,order,ordered,outer,package,packages,page,pairlist,%
      pairs,palette,panel,par,parent,parse,paste,path,pbeta,pbinom,%
      pcauchy,pchisq,pentagamma,persp,pexp,pf,pgamma,pgeom,phyper,pico,%
      pictex,piechart,Platform,plnorm,plogis,plot,pmatch,pmax,pmin,%
      pnbinom,pnchisq,pnorm,points,poisson,poly,polygon,polyroot,pos,%
      postscript,power,ppoints,ppois,predict,preplot,pretty,Primitive,%
      print,prmatrix,proc,prod,profile,proj,prompt,prop,provide,%
      psignrank,ps,pt,ptukey,punif,pweibull,pwilcox,q,qbeta,qbinom,%
      qcauchy,qchisq,qexp,qf,qgamma,qgeom,qhyper,qlnorm,qlogis,qnbinom,%
      qnchisq,qnorm,qpois,qqline,qqnorm,qqplot,qr,Q,qty,qy,qsignrank,%
      qt,qtukey,quantile,quasi,quit,qunif,quote,qweibull,qwilcox,%
      rainbow,range,rank,rbeta,rbind,rbinom,rcauchy,rchisq,Re,read,csv,%
      csv2,fwf,readline,socket,real,Recall,rect,reformulate,regexpr,%
      relevel,remove,rep,repeat,replace,replications,report,require,%
      resid,residuals,restart,return,rev,rexp,rf,rgamma,rgb,rgeom,R,%
      rhyper,rle,rlnorm,rlogis,rm,rnbinom,RNGkind,rnorm,round,row,%
      rownames,rowsum,rpois,rsignrank,rstandard,rstudent,rt,rug,runif,%
      rweibull,rwilcox,sample,sapply,save,scale,scan,scan,screen,sd,se,%
      search,searchpaths,segments,seq,sequence,setdiff,setequal,set,%
      setwd,show,sign,signif,sin,single,sinh,sink,solve,sort,source,%
      spline,splinefun,split,sqrt,stars,start,stat,stem,step,stop,%
      storage,strstrheight,stripplot,strsplit,structure,strwidth,sub,%
      subset,substitute,substr,substring,sum,summary,sunflowerplot,svd,%
      sweep,switch,symbol,symbols,symnum,sys,status,system,t,table,%
      tabulate,tan,tanh,tapply,tempfile,terms,terrain,tetragamma,text,%
      time,title,topo,trace,traceback,transform,tri,trigamma,trunc,try,%
      ts,tsp,typeof,unclass,undebug,undoc,union,unique,uniroot,unix,%
      unlink,unlist,unname,untrace,update,upper,url,UseMethod,var,%
      variable,vector,Version,vi,warning,warnings,weighted,weights,%
      which,while,window,write,\%x\%,x11,X11,xedit,xemacs,xinch,xor,%
      xpdrows,xy,xyinch,yinch,zapsmall,zip},%
   otherkeywords={!,!=,~,$,*,\%,\&,\%/\%,\%*\%,\%\%,<-,<<-,_,/},%
   alsoother={._$},%
   sensitive,%
   morecomment=[l]\#,%
   morestring=[d]",%
   morestring=[d]'% 2001 Robert Denham
  }%

%%%%%%% 
%% Definitions in yellow boxes
\usepackage{etoolbox}
\setbeamercolor{block title}{use=structure,fg=structure.fg,bg=structure.fg!40!bg}
\setbeamercolor{block body}{parent=normal text,use=block title,bg=block title.bg!20!bg}

\BeforeBeginEnvironment{definition}{%
	\setbeamercolor{block title}{fg=black,bg=yellow!20!white}
	\setbeamercolor{block body}{fg=black, bg=yellow!05!white}
}
\AfterEndEnvironment{definition}{
	\setbeamercolor{block title}{use=structure,fg=structure.fg,bg=structure.fg!20!bg}
	\setbeamercolor{block body}{parent=normal text,use=block title,bg=block title.bg!50!bg, fg=black}
}
\BeforeBeginEnvironment{importanttheorem}{%
	\setbeamercolor{block title}{fg=black,bg=red!20!white}
	\setbeamercolor{block body}{fg=black, bg=red!05!white}
}
\AfterEndEnvironment{importanttheorem}{
	\setbeamercolor{block title}{use=structure,fg=structure.fg,bg=structure.fg!20!bg}
	\setbeamercolor{block body}{parent=normal text,use=block title,bg=block title.bg!50!bg, fg=black}
}
\BeforeBeginEnvironment{importantproperty}{%
	\setbeamercolor{block title}{fg=black,bg=red!50!white}
	\setbeamercolor{block body}{fg=black, bg=red!30!white}
}
\AfterEndEnvironment{importantproperty}{
	\setbeamercolor{block title}{use=structure,fg=structure.fg,bg=structure.fg!20!bg}
	\setbeamercolor{block body}{parent=normal text,use=block title,bg=block title.bg!50!bg, fg=black}
}

% Colour for the outline page
\definecolor{outline_colour}{RGB}{230,165,83}
%% Colours for sections, subsections aand subsubsections
\definecolor{section_colour}{RGB}{27,46,28}
\definecolor{subsection_colour}{RGB}{52,128,56}
\definecolor{subsubsection_colour}{RGB}{150,224,154}
% Beginning of a section
\AtBeginSection[]{
	{
		\setbeamercolor{background canvas}{bg=section_colour}
		\begin{frame}[noframenumbering,plain]
			\framesubtitle{\nameofthepart Chapter \insertromanpartnumber \ -- \iteminsert{\insertpart}}
			\tableofcontents[
				currentsection,
				sectionstyle=show/shaded,
				subsectionstyle=show/hide/hide,
				subsubsectionstyle=hide/hide/hide]
		\end{frame}
	\addtocounter{page}{-1}
	%\addtocounter{framenumber}{-1} 
	}
}

% Beginning of a section
\AtBeginSubsection[]{
	{
		\setbeamercolor{background canvas}{bg=subsection_colour}
		\begin{frame}[noframenumbering,plain]
				\framesubtitle{\nameofthepart Chapter \insertromanpartnumber \ -- \iteminsert{\insertpart}}
				\tableofcontents[
					currentsection,
					sectionstyle=show/hide,
					currentsubsection,
					subsectionstyle=show/shaded/hide,
					subsubsectionstyle=show/hide/hide]
			\end{frame}
		\addtocounter{page}{-1}
	}
}

% Beginning of a section
\AtBeginSubsubsection[]{
	{
		\setbeamercolor{background canvas}{bg=subsubsection_colour}
		\begin{frame}[noframenumbering,plain]
				\framesubtitle{\nameofthepart Chapter \insertromanpartnumber \ -- \iteminsert{\insertpart}}
				\tableofcontents[
					currentsection,
					sectionstyle=show/hide,
					subsectionstyle=show/shaded/hide]
					%currentsubsubsection]
			\end{frame}
		\addtocounter{page}{-1}
	}
}


\title{Agent-based and network models}
\subtitle{Lecture 06}
\author{Julien Arino}
\date{September 2023}


\begin{document}
\input{lecture-06-agent-based-and-network-models-concordance}


% The title page
\begin{frame}[noframenumbering,plain]
  \titlepage
\end{frame}
\addtocounter{page}{-1}

\begin{frame}
    \tableofcontents[hideallsubsections]
\end{frame}
\addtocounter{page}{-1}

%%%%%%%%%%%%%%%%%%%
%%%%%%%%%%%%%%%%%%%
%%%%%%%%%%%%%%%%%%%
%%%%%%%%%%%%%%%%%%%
%%%%%%%%%%%%%%%%%%%
%%%%%%%%%%%%%%%%%%%
%%%%%%%%%%%%%%%%%%%
%%%%%%%%%%%%%%%%%%%
\section{Agent-based models (ABM)}

%%%%%%%%%%%%%%%%%%%
%%%%%%%%%%%%%%%%%%%
\subsection{What are agent-based models}

\begin{frame}{ABM $\neq$ IBM}
Early in the life of these models, they were called IBM (individual-based models)
\vfill
Over the years, a ``philosophical'' distinction has emerged:
\begin{itemize}
\item IBM are mathematical models that consider individuals as the units; e.g., DTMC, CTMC, branching processes, etc.
\item ABM are computational models whose study is, for the most part, only possible numerically 
\end{itemize}
\end{frame} 

\begin{frame}{ABM vs Network models}
Network models endow vertices with simple systems and couple them through graphs
\vfill
Can be ABM, but some networks can also be studied analytically
\end{frame} 


%%%%%%%%%%%%%%%%%%%
%%%%%%%%%%%%%%%%%%%
\subsection{When to use ABM}

\begin{frame}{ABM are very useful to decipher contact processes}
Classic mathematical models capture contact by using approximations of what contact could be like
\vfill
Classic models allow some flexibility (see section about incidence functions in Lecture X but they remain limited
\vfill
ABM can model actual trajectories of individuals, so given a definition of what a contact is (how close do you need to be for a contact to take place), can count them efficaciously
\end{frame} 

\begin{frame}{ABM are very useful to understand behavioural responses}
\end{frame} 

%%%%%%%%%%%%%%%%%%%
%%%%%%%%%%%%%%%%%%%
\subsection{When not to use ABM}

\begin{frame}{As with \emph{all} tools, beware!}
There is a law of large numbers effects happening often: if you have many units, unless some emergent behaviour arises, you get the same results using ODEs...
\vfill
With this specific tool, beware!
\vfill
There is a certain tendency in CS people to create \emph{yet another} system and seek \emph{adoption} by users
\end{frame}


%%%%%%%%%%%%%%%%%%%
%%%%%%%%%%%%%%%%%%%
\subsection{Some examples}

\begin{frame}{Antibiotic resistance in hospitals}
D’Agata, Magal, Olivier, Ruan \& Webb. \href{https://doi.org/10.1016/j.jtbi.2007.08.011}{Modeling antibiotic resistance in hospitals: The impact of minimizing treatment duration}, Journal of Theoretical Biology (2007)
\end{frame} 


\begin{frame}{An IBM that's almost an ABM}
This work is a good illustration of the ``cultural proximity'' between IBM and ABM
\vfill
Model is stochastic and individual-based, in good enough form that approximating ODE can be derived
\vfill
Allows for very specific tracking of the status of individuals through the process (almost an ABM in this sense)
\end{frame} 


\begin{frame}{The setup}
Three processes:
\begin{enumerate}
\item admission and exit of patients
\item infection of patients by HCW (health care workers) 
\item contamination of HCW by patients
\end{enumerate}

\vfill
Contamination of HCW is "transient": they are carriers, if they wash their hands properly, they become OK
\vfill
Each day has 3 shifts of 8h for HCW
\vfill
Patients are put in contact by visits of HCW
\vfill
Rules for contaminations per unit time
\end{frame} 


% \begin{frame}
% 
% ![bg contain](https://raw.githubusercontent.com/julien-arino/3MC-course-epidemiological-modelling/main/FIGS//Dagata_etal_patients_profiles.jpg)
% 
% <!-- Patient–HCW contact diagram for four patients and one HCW during one shift. Patient status: uninfected (green), infected with the non-resistant strain (yellow), infected with the resistant strain (red). HCW status: uncontaminated (______ ), contaminated with the non-resistant strain (………), contaminated with the resistant strain (- - - - - ) -->
% 
% \end{frame} 
% 
% 
% \begin{frame}
% 
% ![bg contain](https://raw.githubusercontent.com/julien-arino/3MC-course-epidemiological-modelling/main/FIGS//Dagata_etal_comparisons.jpg)
% 
% <!-- The left (respectively the right) figure corresponds to 1 trajectory (respectively the average over 80 trajectories) of the IBM during one shift, with no exit and admission of patients, and no changes in the infection status of patients -->
% 
% \end{frame} 
% 
% 
% \begin{frame}{Effectiveness of contact tracing}
% 
% Tian, Osgood, Al-Azem & Hoeppner. [Evaluating the Effectiveness of Contact Tracing on Tuberculosis Outcomes in Saskatchewan Using Individual-Based Modeling](https://doi-org.uml.idm.oclc.org/10.1177%2F1090198113493910), Health Education & Behavior (2013)
% \end{frame} 
% 
% 
% \begin{frame}
% Evaluation of contact tracing in TB
% 
% 
% \end{frame} 
% 
% \begin{frame}
% 
% ![bg contain](https://raw.githubusercontent.com/julien-arino/3MC-course-epidemiological-modelling/main/FIGS//TianOsgood_etal_TB_CT.jpeg)
% \end{frame} 
% 
% \begin{frame}
% ![bg contain](https://raw.githubusercontent.com/julien-arino/3MC-course-epidemiological-modelling/main/FIGS//TianOsgood_etal_state_flow_agent.jpeg)
% \end{frame} 
% 
% 
% \begin{frame}
% ![bg contain](https://raw.githubusercontent.com/julien-arino/3MC-course-epidemiological-modelling/main/FIGS//TianOsgood_etal_model_CT.jpeg)
% \end{frame} 
% 
% 
% 
% \begin{frame}
% They can then formulate scenarios
% 
% ![height:60%](https://raw.githubusercontent.com/julien-arino/3MC-course-epidemiological-modelling/main/FIGS//TianOsgood_etal_scenarios.jpeg)
% 
% They then run these scenarios and compare results
% 
% \end{frame} 
% 
% 
% \begin{frame}
% ![bg contain](https://raw.githubusercontent.com/julien-arino/3MC-course-epidemiological-modelling/main/FIGS//TianOsgood_etal_scenario_results.jpeg)
% \end{frame} 
% 
% 
% \begin{frame}{Contacts during Hajj}
% Tofighi, Asgary, Tofighi, Najafabadi, Arino, Amiche, Rahman, McCarthy, Bragazzi, Thommes,  Coudeville, Grunnill, Bourouiba and Wu. [Estimating Social Contacts in Mass Gatherings for Disease Outbreak Prevention and Management (Case of Hajj Pilgrimage)](http://dx.doi.org/10.2139/ssrn.3678581), Tropical Diseases, Travel Medicine and Vaccines
% \end{frame} 
% 
% \begin{frame}{Contacts during Hajj}
% 
% - In a mass gathering event like Hajj, lots of people come together originating from many countries
% - So if propagation occurs during the event, this has the capacity to spread infection far and wide when individuals (pilgrims here) return home
% - Contacts during part of the event are really specific in their configuration
% 
% <div style = "position: relative; bottom: -20%; font-size:20px;">
% 
% Tofighi, Asgary, Tofighi, Najafabadi, Arino, Amiche, Rahman, McCarthy, Bragazzi, Thommes,  Coudeville, Grunnill, Bourouiba and Wu. [Estimating Social Contacts in Mass Gatherings for Disease Outbreak Prevention and Management (Case of Hajj Pilgrimage)](http://dx.doi.org/10.2139/ssrn.3678581), Tropical Diseases, Travel Medicine and Vaccines
% </div>
% 
% \end{frame} 
% 
% 
% \begin{frame}
% 
%  The setup
% 
% - Word of warning: I am quite fuzzy on the specifics :)
% - Pilgrims enter Masjid al-Haram mosque through several gates
% - Proceed to Mataaf (area around Kaaba), circle the Kaaba 7 times counterclockwise (process is the *Tawaf*)
% - Then do seven trips between Safa and Marwah (process is the *Sa'ee*)
% 
% 
% \end{frame} 
% 
% 
% \begin{frame}
% 
% ![bg contain](https://upload.wikimedia.org/wikipedia/commons/thumb/7/7e/Great_Mosque_of_Mecca.jpg/1280px-Great_Mosque_of_Mecca.jpg)
% 
% \end{frame} 
% 
% 
% \begin{frame}
% 
%  Tawaf in pre-COVID-19 times
% 
% <iframe width="800" height="450" src="https://www.youtube.com/embed/L-YyR1oN66w" title="YouTube video player" frameborder="0" allow="accelerometer; autoplay; clipboard-write; encrypted-media; gyroscope; picture-in-picture" allowfullscreen></iframe>
% 
% \end{frame} 
% 
% 
% \begin{frame}
% 
%  Tawaf - Socially distanced version
% 
% <iframe width="800" height="450" src="https://www.youtube.com/embed/Rl8a0wQePCo" title="YouTube video player" frameborder="0" allow="accelerometer; autoplay; clipboard-write; encrypted-media; gyroscope; picture-in-picture" allowfullscreen></iframe>
% 
% \end{frame} 
% 
% 
% \begin{frame}
% 
%  Sa'ee in pre-COVID-19 times
% 
% <iframe width="800" height="450" src="https://www.youtube.com/embed/r1qM-mHj2d0" title="YouTube video player" frameborder="0" allow="accelerometer; autoplay; clipboard-write; encrypted-media; gyroscope; picture-in-picture" allowfullscreen></iframe>
% 
% \end{frame} 
% 
% 
% \begin{frame}
% 
%  Sa'ee - Socially distanced version
% 
% <iframe width="800" height="450" src="https://www.youtube.com/embed/JVges7Q2Mow" title="YouTube video player" frameborder="0" allow="accelerometer; autoplay; clipboard-write; encrypted-media; gyroscope; picture-in-picture" allowfullscreen></iframe>
% 
% \end{frame} 
% 
% 
% \begin{frame}
% 
% - As you can gather from this:
%   - Typically high density crowds
%   - Very specific mixing patterns
% - Opportunities for transmission are very high
% - However, control mechanisms are also available
% 
% $\implies$ understanding contact patterns and frequency would help
% 
% \end{frame} 
% 
% 
% \begin{frame}
% 
% ![bg contain](https://raw.githubusercontent.com/julien-arino/3MC-course-epidemiological-modelling/main/FIGS//ABM_Hajj_MAH_3Dmodel.png)
% 
% \end{frame} 
% 
% 
% \begin{frame}
% 
% ![bg contain](https://raw.githubusercontent.com/julien-arino/3MC-course-epidemiological-modelling/main/FIGS//ABM_Hajj_setup.png)
% \end{frame} 
% 
% 
% \begin{frame}
% 
% ![bg contain](https://raw.githubusercontent.com/julien-arino/3MC-course-epidemiological-modelling/main/FIGS//ABM_Hajj_config_tawaf.png)
% 
% \end{frame} 
% 
% 
% \begin{frame}
% 
% <iframe width="800" height="450" src="https://www.youtube.com/embed/_oOf4uNIghw?loop=1&modestbranding=1&origin=https://julien-arino.github.io/&rel=0" title="YouTube video player" frameborder="0" allow="accelerometer; autoplay; clipboard-write; encrypted-media; gyroscope; picture-in-picture; loop" allowfullscreen>
% </iframe>
% 
% \end{frame} 
% 
% 
% \begin{frame}
% 
% <iframe width="800" height="450" src="https://www.youtube.com/embed/qcWBi17qKnU?start=9&loop=1&modestbranding=1&origin=https://julien-arino.github.io/&rel=0" title="YouTube video player" frameborder="0" allow="accelerometer; autoplay; clipboard-write; encrypted-media; gyroscope; picture-in-picture; loop" allowfullscreen>
% </iframe>


%%%%%%%%%%%%%%%%%%%%%%%%%%%%%
%%%%%%%%%%%%%%%%%%%%%%%%%%%%%
%%%%%%%%%%%%%%%%%%%%%%%%%%%%%
%%%%%%%%%%%%%%%%%%%%%%%%%%%%%
\section{Network models}
%%%%%%%%%%%%%%%%%%%%%%%%%%%%%
%%%%%%%%%%%%%%%%%%%%%%%%%%%%%
\subsection{Why use network models}

% \begin{frame}
% 
%  Comprendre les processus de contact
% 
% - Les modèles classiques permettent une certaine flexibilité (voir par exemple la section dans le [Cours 04](https://julien-arino.github.io/petit-cours-epidemio-mathematique/cours-04-modeles-SIS-SIR.html) sur les fonctions d'incidence ou le [Cours 11](https://julien-arino.github.io/petit-cours-epidemio-mathematique/cours-11-modeles-groupes.html) sur les modèles de groupes), mais cela reste limité et une approximation
% - De la même manière que les modèles agents ([Cours 18](https://julien-arino.github.io/petit-cours-epidemio-mathematique/cours-18-modeles-agents.html)), les modèles en réseaux sont considérés pour rendre plus réaliste la description des processus de contact dans la transmission des agents pathogènes
% 
% \end{frame} 
% 
% 
% \begin{frame}
% 
%  La vie humaine s'organise en réseaux
% 
% - Famille
% - Cercles d'amis
% - Réseau professionel
% - ...
% - Théorie des réseaux sociaux existe et est utilisée depuis des années, par exemple dans le cadre professionnel (e.g., comment fluidifier les interactions dans une entreprise)
% 
% \end{frame} 
% 
% \subsection{Les réseaux sociaux}
% 
% \begin{frame}
% 
% - Avant de considérer des épidémies dans des réseaux, quelques notions de réseaux sociaux, parce que c'est utile de façon générale pour comprendre les réseaux
% - Les méthodes en réseaux sociaux introduisent des mesures qui permettent d'évaluer certaines propriétés des graphes et qui sont utiles à connaître
% - Un réseau est un graphe (mathématique), orienté ou non, dans lequel les arcs représentent les connections (quelles qu'elles soient) entre les individus, qui sont les nœuds du graphe
% 
% \end{frame} 
% 
% 
% 
% \begin{frame}
% 
%  Contexte
% 
% - $\mathcal{G}(\mathcal{V},\mathcal{E})$ un graphe non orienté
% - $\mathcal{D}(\mathcal{V},\mathcal{A})$ un digraphe (graphe orienté)
% - $\mathcal{V}$ l'ensemble des nœuds (*vertices* en anglais)
% - $\mathcal{E}$ l'ensemble des arcs dans le cas non orienté (*edges* en anglais)
% - $\mathcal{A}$ l'ensemble des arcs dans le graphe orienté (*arcs* en anglais)
% 
% \end{frame} 
% 
% 
% \begin{frame}
% 
%  Exemple du réseau de transport aérien
% 
% - Je vais illustrer avec des données du réseau de transport aérien mondial
% - Données assez bonnes (très bonnes parfois), et un avantage flagrant:
%   - Quand un avion part de quelque part et arrive ailleurs, c'est quelque chose d'assez .. déterministe
% 
% \end{frame} 
% 
% 
% \begin{frame}
% 
% ![bg contain](https://raw.githubusercontent.com/julien-arino/petit-cours-epidemio-mathematique/main/FIGS/world_graph-degree.png)
% 
% 
% \end{frame} 
% 
% 
% \begin{frame}
% 
% ![bg contain](https://raw.githubusercontent.com/julien-arino/petit-cours-epidemio-mathematique/main/FIGS/Manitoba_network_schema_planar_oriented.png)
% 
% 
% \end{frame} \begin{frame}
% 
%  Densité du graphe
% 
% Un graph (resp. digraphe) est **complet** si toute paire de nœuds est connecté (resp. est connecté par un arc dans chaque direction)
% 
% S'il y a $n=|\mathcal{V}|$ nœeuds dans le graphe, alors il y a $n(n-1)/2$ (resp. $n(n-1)$) arcs dans le graphe (resp. digraphe) complet
% 
% (On ne compte pas les connections d'un nœud sur lui même)
% 
% Densité de $\mathcal{G}$ (graphe non orienté)
% $$
% \mathsf{dens}_\mathcal{G}=\frac{2\ |\mathcal{E}|}{n(n-1)}
% $$
% Densité de $\mathcal{D}$ (graphe orienté)
% $$
% \mathsf{dens}_\mathcal{D}=\frac{|\mathcal{A}|}{n(n-1)}
% $$
% 
% \end{frame} \begin{frame}
% 
%  Densité des digraphes considérés
% 
% | Digraphe |  nœuds |  arcs | densité |
% |\end{frame} \begin{frame}\end{frame} \begin{frame}\end{frame} \begin{frame}-|:\end{frame} \begin{frame}\end{frame} \begin{frame}-:|:\end{frame} \begin{frame}\end{frame} \begin{frame}:|:\end{frame} \begin{frame}\end{frame} \begin{frame}-:|
% | Manitoba | 24 | 64 | 0.1159 |
% | Canada | 222 | 804 | 0.0164 |
% | Amérique du Nord | 934 | 7,814 | 0.009 |
% | Global | 3403 | 32,576 | 0.0028 |
% 
% \end{frame} \begin{frame}
% 
%  Degré
% 
% **Degré** $d_\mathcal{G}(v)$ du nœud $v\in\mathcal{V}$ dans $\mathcal{G}$: nombre d'arcs incidents à $v$
% 
% **Degré entrant** $d^-_\mathcal{D}(v)$ du nœud $v\in\mathcal{V}$ dans $\mathcal{D}$: nombre d'arcs avec tête $v$
% 
% **Degré sortant** $d^+_\mathcal{D}(v)$ du nœud $v\in\mathcal{V}$ dans $\mathcal{D}$: nombre d'arcs avec queue $v$
% 
% **Degré** $d_\mathcal{D}(v)$ du nœud $v\in\mathcal{V}$ dans $\mathcal{D}(\mathcal{V},\mathcal{A})$: nombre d'arcs incidents à $v$ dans le graphe non orienté sous-jacent $\mathcal{G}$ de $\mathcal{D}$ (où tout arc est considérée comme un arc "bidirectionnel")
% 
% \end{frame} \begin{frame}
% 
%  Degré entrant global du réseau de transport aérien
% 
% | Ville | Pays | Degré entrant | Rang |
% |\end{frame} \begin{frame}\end{frame} \begin{frame}|\end{frame} \begin{frame}\end{frame} \begin{frame}\end{frame} \begin{frame}|:\end{frame} \begin{frame}\end{frame} \begin{frame}\end{frame} \begin{frame}-:|:\end{frame} \begin{frame}-:| 
% | Londres | GB | 365 | 1 |
% | Paris | France | 294 | 2 |
% | Frankfurt | Allemagne | 287 | 3 |
% | Atlanta | USA | 249 | 4 |
% | New York | USA | 241 | 5 |
% | Moscou | Russie | 225 | 6 |
% | Amsterdam | Pays-Bas | 204 | 7 |
% | Chicago | USA | 203 | 8 |
% | Munich | Allemagne | 200 | 9 |
% | Milan | Italie | 181 | 10 |
% 
% \end{frame} \begin{frame}
% 
%  Le degré change pendant l'année 
% 
% Les graphes sont dynamiques !
% 
% ![bg right:72%](https://raw.githubusercontent.com/julien-arino/petit-cours-epidemio-mathematique/main/FIGS/IATA_outdegree_YEA_2005_to_2010.png)
% 
% 
% \end{frame} \begin{frame}
% 
%  Plus court chemin
% 
% Soit $\mathcal{D}$ un digraphe. Le (ou les) plus court(s) chemin(s) de $i$ à $j$ dans $\mathcal{V}$:
% $$
% d_\mathcal{D}(i,j)=\min_{p\in\mathcal{P}(i,j)} f(p)
% $$
% où $\mathcal{P}(i,j)$ est l'ensemble des chemins de $i$ à $j$ et $f(p)$ est un valuation des arcs dans le chemin $p$. On définit $d_\mathcal{D}(i,j)=\infty$ s'il n'existe pas de chemin de $i$ à $j$ 
% 
% $f(p)$ peut être
% - le nombre d'arcs dans $p$ de $i$ à $j$ (**distance géodésique**)
% - Distance du grand cercle des arcs de $p$
% - durée des vols des arcs de $p$
% 
% \end{frame} \begin{frame}
% 
%  Excentricité
% 
% **Excentricité** (ou **nombre de Köonig**) du nœud $v\in\mathcal{V}$ dans $\mathcal{G}(\mathcal{V},\mathcal{E})$
% $$
% e(v)=\max_{v'\in\mathcal{V}}d_\mathcal{D}(v,v')
% $$
% **Excentricité entrante** du nœud $v\in\mathcal{V}$ dans $\mathcal{D}(\mathcal{V},\mathcal{A})$
% $$
% e^-(v)=\max_{v'\in\mathcal{V}}d_\mathcal{D}(v',v)
% $$
% **Excentricité sortante** du nœud $v\in\mathcal{V}$ dans $\mathcal{D}(\mathcal{V},\mathcal{A})$
% $$
% e^+(v)=\max_{v'\in\mathcal{V}}d_\mathcal{D}(v,v')
% $$
% 
% \end{frame} \begin{frame}
% 
% | Graphe | $e^-(YWG)$ | $e^+(YWG)$ |
% |\end{frame} \begin{frame}\end{frame} \begin{frame}--|\end{frame} \begin{frame}\end{frame} \begin{frame}\end{frame} \begin{frame}\end{frame} \begin{frame}|\end{frame} \begin{frame}\end{frame} \begin{frame}\end{frame} \begin{frame}\end{frame} \begin{frame}|
% | Manitoba | 2 | 3 (Lynn Lake) |
% | Canada | 7 $^{(*)}$ | 7 $^{(*)}$ |
% | Amérique du Nord | 7 $^{(**)}$| 8 (Stony River) |
% | Global | 7 $^{(***)}$ | 8 (Stony River) |
% 
% | <!-- --> | <!-- --> |
% |\end{frame} \begin{frame}|\end{frame} \begin{frame}|
% | ( * ) | Peawanuck (ON), Port Hope Simpson (NL) |
% ( ** ) | ( * ) + Lopez Island, Kwethluk, Chuathbaluk |
% ( *** ) | ( ** ) + Hooker Creek, Birdsville, Beni, Balalae, Thargomindah |
% 
% \end{frame} \begin{frame}
% 
%  Rayon
% 
% **Rayon** de $\mathcal{G}$
% $$
% \rho_\mathcal{G}=\min_{v\in\mathcal{V}}e(v)
% $$
% **Rayon entrant** de $\mathcal{D}$
% $$
% \rho_\mathcal{D}^-=\min_{v\in\mathcal{V}}e^-(v)
% $$
% **Rayon sortant** de $\mathcal{D}$
% $$
% \rho_\mathcal{D}^+=\min_{v\in\mathcal{V}}e^+(v)
% $$
% 
% rayon = $\min(\max(\cdot))$ $\rightarrow$ directionalité
% 
% \end{frame} \begin{frame}
% 
%  Rayon du réseau de transport aérien
% 
% | Graphe | $\rho^-$ | $\rho^+$ |
% |\end{frame} \begin{frame}\end{frame} \begin{frame}--|\end{frame} \begin{frame}\end{frame} \begin{frame}\end{frame} \begin{frame}-|\end{frame} \begin{frame}\end{frame} \begin{frame}\end{frame} \begin{frame}-|
% | Manitoba | 2 | 3 |
% | Canada | 6 | 6 |
% | Amérique du Nord | 6 | 7 |
% | Global | 7 | 7 |
% 
% 
% \end{frame} \begin{frame}
% 
%  Centre d'un graphe
% 
% **Centre** de $\mathcal{D}$:
% $$
% \mathcal{C}_\mathcal{D}=\left\{v\in\mathcal{V}:e(v)=\rho_\mathcal{D}\right\}
% $$
% 
% \end{frame} \begin{frame}
% 
%  Centre du réseau de transport aérien
% 
% | Graphe | $\mathcal{C}^-$ | $\\\|\mathcal{C}^-\\\|$ | $\mathcal{C}^+$ | $\\\|\mathcal{C}^+\\\|$ |
% |\end{frame} \begin{frame}|\end{frame} \begin{frame}|\end{frame} \begin{frame}|\end{frame} \begin{frame}|\end{frame} \begin{frame}|
% | Manitoba | 2 | 1 (YWG) | 3 | 7 |
% | Canada | 6 | 1 (YTO) | 6 | 1 (YTO) |
% | Amérique du Nord | 6 | 1 (YTO) | 7 | 18 |
% | Global | 7 | 131 | 7 | 20 |
% 
% $\{$YYC,YEA,Halifax,Kelowna,Moncton,YMQ,YOW,Quebec,St John's,YTO,YVR, Victoria,YWG$\}\subset\mathcal{C}^-$
% 
% $\{$Toronto,Vancouver$\}\subset\mathcal{C}^+$
% 
% \end{frame} \begin{frame}
% 
%  Diamètre
% 
% **Diamètre** de $\mathcal{D}$
% $$
% \mathsf{diam}_\mathcal{D}=\max_{v\in\mathcal{V}}e(v)
% $$
% 
% diamètre = max(max(.)) $\rightarrow$ pas de directionalité
% 
% \end{frame} \begin{frame}
% 
% 
% Graphe | Diamètre |
% |\end{frame} \begin{frame}\end{frame} \begin{frame}|:\end{frame} \begin{frame}\end{frame} \begin{frame}--:|
% | Manitoba | 5 |
% | Canada | 12 |
% | Amérique du Nord | 13 |
% | Global | 13 |
% 
% 
% \end{frame} \begin{frame}
% 
%  Péripherie d'un graphe
% 
% **Péripherie** de $\mathcal{D}$ 
% $$
% \mathcal{P}_\mathcal{D}=\left\{v\in\mathcal{V}:e(v)=\mathsf{diam}_\mathcal{D}\right\}
% $$
% 
% \end{frame} \begin{frame}
% 
% | Graphe | Péripherie entrante | Péripherie sortante |
% |\end{frame} \begin{frame}\end{frame} \begin{frame}--|\end{frame} \begin{frame}\end{frame} \begin{frame}\end{frame} \begin{frame}\end{frame} \begin{frame}\end{frame} \begin{frame}\end{frame} \begin{frame}\end{frame} \begin{frame}|\end{frame} \begin{frame}\end{frame} \begin{frame}\end{frame} \begin{frame}\end{frame} \begin{frame}\end{frame} \begin{frame}\end{frame} \begin{frame}\end{frame} \begin{frame}|
% | Manitoba | Lynn Lake | Cross Lake, Red Sucker Lake, Brandon |
% | Canada | Peawanuck | Peawanuck, Port Hope Simpson | Port Hope Simpson |
% | Amérique du Nord | Stony River | Peawanuck, Port Hope Simpson |
% | Global | Stony River, Hooker Creek, Peawanuck | Hooker Creek, Beni, Peawanuck, Port Hope Simpson |
% 
% \end{frame} \begin{frame}
% 
%  Bien d'autres mesures
% 
% - betweenness
% - closeness
% - $k$-cores
% - $\ldots$
% 
% \end{frame} \begin{frame}
% 
% <!-- _backgroundImage: "linear-gradient(to bottom, #f1c40f, 20%, white)" -->
%  <!--fit-->Cadre général des modèles en réseaux
% 
% \end{frame} \begin{frame}
% 
% - Voir par exemple 
%   - Newman. [Spread of epidemic disease on networks](https://doi.org/10.1103/PhysRevE.66.016128), 2002
%   - Keeling & Eames. [Networks and epidemic models](https://doi.org/10.1098/rsif.2005.0051), 2005
%   - Meyers, Pourbohloul, Newman, Skowronski & Brunham. [Network theory and SARS: predicting outbreak diversity](https://doi.org/10.1016/j.jtbi.2004.07.026), 2005
%   - Meyers, Newman & Pourohloul. [Predicting epidemics on directed contact networks](https://doi.org/10.1016/j.jtbi.2005.10.004), 2006
%   - Bansal, Read, Pourbohloul & Meyers. [The dynamic nature of contact networks in infectious disease epidemiology](https://doi.org/10.1080/17513758.2010.503376), 2010 
% 
% 
% \end{frame} \begin{frame}
% 
% - Typiquement, on considère un graphe (ou digraphe selon les cas) dans lequel:
%   - chaque nœud est un individu 
%   - l'existence d'un arc de $i$ vers $j$ indique que $i$ est en contact avec $j$ et peut lui transmettre l'infection
%   - dans le cas non orienté, l'existence d'un arc de $i$ vers $j$ implique celle d'un arc (le même) de $j$ vers $i$ et établit que les deux individus sont connectés
% - La connexion n'est pas permanente, mais décrit plutôt la possibilité d'une connexion: $i$ et $j$ entrent en contact de façon régulière
% 
% \end{frame} \begin{frame}
% 
%  Matrice d'adjacence
% 
% On utilisera souvent la **matrice d'adjacence** $A=[a_{ij}]$, dans laquelle $a_{ij}=1$ si le nœud $i$ a un lien vers le nœud $j$ et $a_{ij}=0$ sinon
% 
% On écrit parfois $A(\mathcal{D})$ pour indiquer que $A$ est la matrice d'adjacence du digraphe $\mathcal{D}$, et dans l'autre sens, $\mathcal{D}(A)$ pour indiquer que le graphe est construit en utilisant la matrice d'adjacence
% 
% Si le graphe est non orienté, alors $A$ est symmétrique
% 
% \end{frame} \begin{frame}
% 
%  Nature du réseau
% 
% - Parfois on dispose de données précises sur les liens entre individus (sondages, etc.)
% - Souvent on idéalise des réseaux, on choisit des réseaux avec des propriétés données
% 
% \end{frame} \begin{frame}
% 
%  La distribution des degrés du (di)graphe
% 
% La **transmissibilité** $T$ d'une maladie dans un graphe est la probabilité moyenne qu'un individu infectieux transmette la maladie à un individu susceptible avec qui il/elle est en contact
% 
% Dans un réseau non corrélé,
% $$
% T_c = \frac{\langle k\rangle}{\langle k^2\rangle-\langle k\rangle}
% $$
% où $\langle k\rangle$ et $\langle k^2\rangle$ sont le degré moyen et la moyenne du carré du degré
% 
% Il est nécessaire que $T>T_c$ pour qu'un *outbreak* devienne une épidémie majeure
% 
% \end{frame} \begin{frame}
% 
% <!-- _backgroundImage: "linear-gradient(to bottom, #f1c40f, 20%, white)" -->
%  <!--fit-->La librairie EpiModel
% 
% <div style = "text-align: justify; position: relative; bottom: -35%; font-size:18px;">
% 
% Jenness SM, Goodreau SM and Morris M. [EpiModel: An R Package for Mathematical Modeling of Infectious Disease over Networks](https://doi.org/10.18637%2Fjss.v084.i08). Journal of Statistical Software. 2018; 84(8): 1-47
% </div>
% \end{frame} 


\begin{frame}{EpiModel}
\code{R} library providing tools to simulate and analyse network epidemiological models
\vfill
Provides two types of approaches
\begin{itemize}
\item Simulation of ODE compartmental models (not so interesting)
\item Simulation of network models
\end{itemize}
\vfill
Their \href{https://www.epimodel.org}{website} has several useful tutorials
\vfill
Part of the \href{http://statnet.org/}{statnet} meta-library
\end{frame}

\end{document}
